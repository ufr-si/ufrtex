% ---
% Este é o documento principal do template para Trabalhos Acadêmicos da Universidade Federal de Rondonópolis. Esse template usa o AbnTeX2 , que possui diversos comandos para elaboração de um documento acadêmico.
% ---
%% abtex2-modelo-trabalho-academico.tex, v-1.9.7 laurocesar
%% Copyright 2012-2018 by abnTeX2 group at http://www.abntex.net.br/ 
%%
%% This work may be distributed and/or modified under the
%% conditions of the LaTeX Project Public License, either version 1.3
%% of this license or (at your option) any later version.
%% The latest version of this license is in
%%   http://www.latex-project.org/lppl.txt
%% and version 1.3 or later is part of all distributions of LaTeX
%% version 2005/12/01 or later.
%%
%% This work has the LPPL maintenance status `maintained'.
%% 
%% The Current Maintainer of this work is the abnTeX2 team, led
%% by Lauro César Araujo. Further information are available on 
%% http://www.abntex.net.br/
%%
%% This work consists of the files abntex2-modelo-trabalho-academico.tex,
%% abntex2-modelo-include-comandos and abntex2-modelo-references.bib
%%
% ------------------------------------------------------------------------

% abnTeX2: Modelo de Trabalho Academico (tese de doutorado, dissertacao de
% mestrado e trabalhos monograficos em geral) em conformidade com 
% ABNT NBR 14724:2011: Informacao e documentacao - Trabalhos academicos -
% Apresentacao 
% --------------------------------------------
\documentclass[
	% -- opções da classe memoir --
	12pt,				% tamanho da fonte
	openright,			% capítulos começam em pág ímpar (insere página vazia caso preciso)
	oneside,			% para impressão frente. Oposto a twoside
	a4paper,			% tamanho do papel. 
	% -- opções da classe abntex2 --
	chapter=TITLE,		% títulos de capítulos convertidos em letras maiúsculas
	% section=TITLE,		% títulos de seções convertidos em letras maiúsculas
	%subsection=title,	% títulos de subseções convertidos em letras maiúsculas
	%subsubsection=TITLE,% títulos de subsubseções convertidos em letras maiúsculas
    % sumario = tradicional,
	% -- opções do pacote babel --
	english,			% idioma adicional para hifenização
	french,				% idioma adicional para hifenização
	spanish,			% idioma adicional para hifenização
	brazil				% o último idioma é o principal do documento
	]{abntex2}
\usepackage{ufr/ufr} % usando package da ufr.



% ---
% compila o indice
% ---
\makeindex
% ---

% ----
% Início do documento
% ----
\begin{document}

% Seleciona o idioma do documento (conforme pacotes do babel)
%\selectlanguage{english}
\selectlanguage{brazil}

% Retira espaço extra obsoleto entre as frases.
\frenchspacing 

% ----------------------------------------------------------
% ELEMENTOS PRÉ-TEXTUAIS
% ----------------------------------------------------------
\pretextual

% ---
% Capa
% ---
\imprimircapa
% ---

% ---
% Folha de rosto
% (o * indica que haverá a ficha bibliográfica)
% ---
\imprimirfolhaderosto*
% ---

% ---
% Inserir a ficha bibliografica
% ---

\input{pre-textual/ficha-bibliografica.tex}
% ---
% Inserir errata
% ---
\input{pre-textual/errata.tex}
% ---
% Inserir folha de aprovação
% ---
% Isto é um exemplo de Folha de aprovação, elemento obrigatório da NBR
% 14724/2011 (seção 4.2.1.3). Você pode utilizar este modelo até a aprovação do trabalho. Após isso, substitua todo o conteúdo deste arquivo por uma imagem da página assinada pela banca com o comando abaixo:
%
% \begin{folhadeaprovacao}
% \includepdf{folhadeaprovacao_final.pdf}
% \end{folhadeaprovacao}

\begin{folhadeaprovacao}

  \begin{center}
    {\bfseries\imprimirautor}

    \vspace*{2.5cm}
    % \vspace*{\fill}
    \begin{center}
      \bfseries\imprimirtitulo
    \end{center}
    \vspace*{2cm}
    
    \hspace{.45\textwidth}
    \begin{minipage}{.5\textwidth}
        \imprimirpreambulo
    \end{minipage}%
    %  \vspace*{\fill}
   \end{center}
    \vspace*{1cm}
    \begin{center}
        Aprovado em: 24 de novembro de 2012.
    \end{center}
    \vspace*{1cm}
    
    
    \begin{center}
        \textbf{Banca Examinadora:}
    \end{center}
    \par
   %TODO arrumar isso aqui em ambiente, e possibilitar até 4 pessoas.
   \assinatura{\textbf{\imprimirorientador} \\ Orientador} 
   \assinatura{\textbf{Coorientador} \\ Convidado2 }
   \assinatura{\textbf{Professor} \\ Convidado 2}
   %\assinatura{\textbf{Professor} \\ Convidado 3}
   %\assinatura{\textbf{Professor} \\ Convidado 4}
      
\begin{center}
    \vspace*{0.5cm}
    {\large\imprimirlocal}
    \par
    {\large\imprimirdata}
    \vspace*{1cm}
  \end{center}  
\end{folhadeaprovacao}

% ---
% Dedicatória (elemento opcional).
% ---
\begin{dedicatoria}
   Dedicatória. Elemento opcional. Frase por meio da qual o autor presta homenagem ou dedica o trabalho a alguém. O título Dedicatória não deve aparecer no início da folha. Se não desejar escrever uma dedicatória, delete este elemento.
\end{dedicatoria}


% --- 
% Agradecimentos (elemento opcional).
% --- 
% ---
% Agradecimentos
% ---
\begin{agradecimentos}

\textit{Elemento opcional. Menção que o autor faz a pessoas e/ou instituições que colaboraram de maneira relevante na elaboração do trabalho. Se não desejar escrever agradecimentos, delete esta página. }

Ao professor fulano, pela...

Ao Senhor cicrano, pela...

A todos que direta ou indiretamente colaboraram na execução deste trabalho.
\end{agradecimentos}
% ---

% ---
% Epígrafe (elemento opcional)
%--- 
% \input{pre-textual/epigrafe.tex}

% ---
% Resumo (elemento OBRIGATÓRIO)
%--- 
% ---
% RESUMOS
% ---
% resumo em português

\begin{resumo}
 De acordo com a ABNT NBR 6028:2021, o resumo informativo deve ressaltar o objetivo, o método, os resultados e as conclusões do documento. Ele deve ser composto de uma sequência de frases concisas e afirmativas. Convém usar o verbo na terceira pessoa do singular. O texto do resumo deve ser digitado em um parágrafo único, justificado. O espaçamento entre linhas é simples e o tamanho da fonte é 12. Deve conter de 150 a 500 palavras. As palavras-chave devem figurar logo abaixo do resumo, antecedidas da expressão Palavras-chave, seguida de dois-pontos, separadas entre si por ponto e vírgula e finalizadas por ponto. Devem ser grafadas com as iniciais em letra minúscula, com exceção dos substantivos próprios e nomes científicos.
 
\textbf{Palavras-chave}: latex; abntex; editoração de texto.

\end{resumo}

% resumo em inglês
\begin{resumo}[Abstract]
 \begin{otherlanguage*}{english}
   This is the english abstract.

   \vspace{\onelineskip}
 
   \noindent 
   \textbf{Keywords}: latex; abntex; text editoration.
 \end{otherlanguage*}
\end{resumo}

% Se quiser, é possível adicionar outros idiomas para resumo, bastando utilizar os exemplos abaixo.  

% % resumo em francês 
% \begin{resumo}[Résumé]
% \SingleSpacing
%  \begin{otherlanguage*}{french}
%     Il s'agit d'un résumé en français.
 
%    \textbf{Mots-clés}: latex. abntex. publication de textes.
%  \end{otherlanguage*}
% \end{resumo}

% % resumo em espanhol
% \SingleSpacing
% \begin{resumo}[Resumen]
%  \begin{otherlanguage*}{spanish}
%    Este es el resumen en español.
  
%    \textbf{Palabras clave}: latex. abntex. publicación de textos.
%  \end{otherlanguage*}
% \end{resumo}
% ---
% ---
% Listas
%--- 
% Aqui vão as listas diversas, na ordem que se encontram. Caso não haja alguma das listas você pode simplesmente eliminá-la removendo o código relacionado à ela.

% ---
% inserir lista de ilustrações
% ---
\pdfbookmark[0]{\listfigurename}{lof}
\listoffigures*
\cleardoublepage
% ---

% ---
% inserir lista de quadros
% ---
\pdfbookmark[0]{\listofquadrosname}{loq}
\listofquadros*
\cleardoublepage
% ---

% ---
% Inserir lista de tabelas
% ---
\pdfbookmark[0]{\listtablename}{lot}
\listoftables*
\cleardoublepage
% ---

% ---
%%% Inserir lista de abreviaturas e siglas %%% 
% Para inserir uma nova abreviatura/sigla, basta usar o formato abaixo
% ---
\begin{siglas}
  \item[ABNT] Associação Brasileira de Normas Técnicas
  \item[abnTeX] ABsurdas Normas para TeX
\end{siglas}
% ---

% ---
%%% Inserir lista de símbolos %%%
% Para inserir uma nova abreviatura/sigla, basta usar o formato abaixo
% ---
\begin{simbolos}
  \item[$ \Gamma $] Letra grega Gama
  \item[$ \Lambda $] Lambda
  \item[$ \zeta $] Letra grega minúscula zeta
  \item[$ \in $] Pertence
\end{simbolos}
% ---

% ---
% Inserção do sumario
% ---
\pdfbookmark[0]{\contentsname}{toc}
\tableofcontents*
\cleardoublepage
% ---
    
% ELEMENTOS TEXTUAIS
% ----------------------------------------------------------
\textual %NÃO REMOVER 
\pagestyle{simple} %NÃO REMOVER 

%adicione aqui as referencias aos capitulos, como abaixo, na ordem que devem aparecer.
%\input{capitulos/nome_arquivo.tex}

% ----------------------------------------------------------
% Introdução (exemplo de capítulo sem numeração, mas presente no Sumário)
% ----------------------------------------------------------
\chapter{Introdução}

A introdução é o primeiro elemento textual e contém alguns itens importantes do projeto de pesquisa: tema, questões de pesquisa, objetivos e justificativa (sucinta). Deve situar o autor da pesquisa em relação ao que irá estudar, apresentando em linhas gerais como chegou ao tema e como pretende desenvolvê-lo em sua pesquisa. Ela deve se encerrar apresentando ao leitor a organização retórica de seu trabalho, ou seja, as partes que compõem o TCC.

\section{Olá mundo}

Segundo Machado, Lousada e Abreu-Tardelli (2005, p. 83), “a introdução pode ser vista como um trailer do que o leitor verá no seu trabalho, nem mais nem menos”. É uma seção que deve levar o leitor a querer ler o trabalho, seduzindo-o.

\subsection{Olá mundo}

Uma dica útil dada pelas autoras é apresentar inicialmente o “objeto” sobre o qual trata a pesquisa em um relato de como você chegou ao tema, quais os motivos mais relevantes, as buscas que efetuou, as decisões tomadas e as teorias que foi selecionando ao longo dessa busca. 

\subsubsection{Olá mundo}

Aqui serão dadas indicações gerais para a apresentação gráfica de seu trabalho, contudo, você pode consultar a NBR 14724:2011 para obter mais informações sobre a apresentação de trabalhos acadêmicos.

\subsubsubsection{Olá mundo}

Todo o texto deve ser digitado em espaço 1,5 cm, exceto: citações de mais de três linhas, notas de rodapé, referências, legendas e fontes das ilustrações e das tabelas que devem ser digitados em espaço simples. As referências, ao final do trabalho, devem ser separadas entre si por um espaço simples em branco.

\begin{citacao}
    Esta é uma citação, \lipsum[20]
\end{citacao}

As margens da página devem ser de 3 cm nas margens esquerda e superior e 2 cm nas margens direita e inferior.

Os títulos dos capítulos (seções primárias, secundárias, etc.) devem ser digitados após a sua numeração (indicação de seção), separados por um espaço. O texto deve iniciar em outra linha, separado por um espaço entrelinhas de 1,5. 

As seções primárias iniciam-se em nova página e são grafadas em caixa alta e negrito. As seções secundárias são grafadas em negrito com apenas a primeira letra maiúscula. As seções terciárias não são grafadas em negrito. Escreva um título criativo consoante o arcabouço teórico e seu plano de trabalho constante no sumário. Utilize junto a este template a NBR 6024:2012 – Numeração progressiva das seções de um documento.

A fonte utilizada no texto é Arial ou Times, tamanho 12, excetuando-se citações com mais de três linhas, notas de rodapé, paginação, legendas e fontes das ilustrações e das tabelas, que devem ser em tamanho 11. 

As citações diretas com mais de três linhas devem ser destacadas com recuo de 4 cm da margem esquerda, fonte tamanho 11 e sem as aspas. Consulte sempre a norma específica para citações, NBR 10520:2002.

As páginas pré-textuais (todas que precedem a Introdução) devem ser contadas, mas não numeradas, exceto a capa e página da Ficha Catalográfica. A numeração deve figurar a partir da Introdução, em algarismos arábicos, no canto superior direito da folha, a 2 cm da borda superior, ficando o último algarismo a 2 cm da borda direita da folha, fonte 11.


% \input{tabelas/tabela1.tex}
\input{capitulos/modelo-trab.tex}
% ---
% Capitulo de revisão de literatura
% ---
\chapter{Lorem ipsum dolor sit amet}
% ---
\cite{godinho2016vida} e \cite{mello2006metodologia} e \citeauthyear{org2001fonoaudiologo}
São autores que estamos citando.

% ---
\section{Aliquam vestibulum fringilla lorem}
% ---

\lipsum[1]

\lipsum[2-3]

\input{capitulos/cap-4.tex}
\input{capitulos/conclusao.tex}


% ----------------------------------------------------------
% ----------------------------------------------------------


% ----------------------------------------------------------
% ELEMENTOS PÓS-TEXTUAIS
% ----------------------------------------------------------
\postextual
% ----------------------------------------------------------
% Referências bibliográficas
% ----------------------------------------------------------
\bibliography{bibliografia}

% ----------------------------------------------------------
% Apêndices
% ----------------------------------------------------------
% Inicia os apêndices
% ---
    \begin{apendicesenv}
        \input{postextual/apendices/apendice-a.tex}
        % ----------------------------------------------------------
\chapter{Nullam elementum urna vel imperdiet}
% ----------------------------------------------------------
\lipsum[55-57]
    \end{apendicesenv}
    % ---

% ----------------------------------------------------------
% Anexos
% ----------------------------------------------------------
    \begin{anexosenv}
        \input{postextual/anexos/anexo-a.tex}
        % ---
\chapter{Cras non urna sed feugiat cum sociis natoque penatibus et 
parturient montes nascetur ridiculus mus}
% ---

\lipsum[31]

    \end{anexosenv}

\end{document}
